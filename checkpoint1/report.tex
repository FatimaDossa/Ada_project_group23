\documentclass[12pt]{article}
\usepackage[a4paper,margin=1in]{geometry}
\usepackage{hyperref}
\usepackage{enumitem}

\title{Paper Selection Proposal - Checkpoint 1 }
\author{Fatima Dossa and Maira Khan \\ Group 23  \\ L3}
\date{\today}

\begin{document}
\maketitle

\section{Paper Details}
\begin{itemize}
    \item \textbf{Title:} Graph Mining Healthcare Approach: Analysis and Recommendation 
    \item \textbf{Authors:} Hiba G. Fareed, Isam A. Alobaidi, Jennifer L. Leopold, Layth M. Almashhadani, Nathan W. Eloe 
    \item \textbf{Conference:} POLIBITS, Vol 66, No 1 (2024)
    \item \textbf{DOI/Link:} \href{https://www.cys.cic.ipn.mx/ojs/index.php/polibits/article/view/4704}{https://www.cys.cic.ipn.mx/ojs/index.php/polibits/article/view/4704}
\end{itemize}

\section{Summary}
The paper discusses the growth of e-health applications and the need for effective predictive techniques. It highlights the limitations of traditional clustering techniques and proposes collaborative graph mining methods to improve prediction accuracy. The study employs frequent subgraph mining (FSM) and discriminative subgraph mining (DSM) techniques to analyze medical data from the MIMIC dataset. Experimental evaluations demonstrate that DSM achieves higher accuracy in predicting patient recovery outcomes compared to FSM.


\section{Justification}
This paper is relevant to our project as it explores predictive analytics in healthcare using advanced graph mining techniques. Its approach addresses key challenges such as medical data errors and the lack of explicit links between patient states and recorded data. The study's emphasis on subgraph mining for recommendation systems aligns with our interest in data-driven decision-making and machine learning applications. Additionally, since we are both interested in healthcare, this paper provides a perfect opportunity to learn more and understand graph algorithms by seeing their applicability in real-life projects.


\section{Implementation Feasibility}
Available resources include:
\begin{itemize}
    \item \textbf{Datasets:} The study uses the MIMIC dataset, which is publicly available and contains critical care data.
    \item \textbf{Theoretical Framework:} The research builds on established graph mining techniques, including frequent subgraph mining and discriminative subgraph mining, making it theoretically replicable.
\end{itemize}
These resources will help in reproducing the results and extending the research.

\section{Team Responsibilities}
The team will divide tasks as follows:
\begin{itemize}
    \item \textbf{Reading and Understanding:} Both will summarize key ideas and methodologies.
    \item \textbf{Coding and Implementation:} Both will work on reproducing experiments and analyzing results.
    \item \textbf{Writing and Documentation:} Both will draft reports, discussions, and presentations.
\end{itemize}

\end{document}
